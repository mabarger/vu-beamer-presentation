% Documentclass und Aspectratio
\documentclass[10pt,aspectratio=1610]{beamer}

% Beamer options and themes
\usepackage{pgfpages}
%\setbeameroption{show notes on second screen=right} 
\usetheme[progressbar=frametitle]{metropolis}
\setbeamertemplate{footline}[frame number]

% Biber
\usepackage{xcoffins}


% Used packages
\usepackage{appendixnumberbeamer}
\usepackage{booktabs}
\usepackage{microtype}
\usepackage[utf8]{inputenc}
\usepackage{csquotes}
\usepackage[scale=2]{ccicons}
\usepackage{svg}
\usepackage{soul}
\usepackage{xcolor}

% custom Colors
\definecolor{vu_blue}{HTML}{0077b3}
\definecolor{special_gray}{HTML}{252525}
\definecolor{special_light_gray}{HTML}{A0A0A0}
\definecolor{vu_orange_red}{HTML}{b33c00}
\definecolor{darkorange}{HTML}{b33c00}
% custom Colors

% Listings
\usepackage{listings}
\usepackage{xcolor}

% Fancy underlines for text to refer to images
\definecolor{greenish}{HTML}{9ec69c}
\newcommand{\ulinegreen}[1]{\setulcolor{greenish}\ul{#1}}

\definecolor{greyish}{HTML}{a1a1a1}
\newcommand{\ulinegrey}[1]{\setulcolor{greyish}\ul{#1}}

% Basic C style
\lstdefinestyle{cstyle}{
    language=C,
    basicstyle=\ttfamily\footnotesize,
    keywordstyle=\color{vu_blue}\bfseries,
    stringstyle=\color{red},
    commentstyle=\color{gray}\itshape,
    numberstyle=\tiny\color{darkgray},
    numbers=none,
    stepnumber=1,
    showstringspaces=false,
    tabsize=4,
    breaklines=true,
    breakatwhitespace=false,
    frame=single,
    backgroundcolor=\color{white},
    morekeywords={size_t},
    literate={0}{{\textcolor{vu_orange_red}{0}}}{1}
             {1}{{\textcolor{vu_orange_red}{1}}}{1}
             {2}{{\textcolor{vu_orange_red}{2}}}{1}
             {3}{{\textcolor{vu_orange_red}{3}}}{1}
             {4}{{\textcolor{vu_orange_red}{4}}}{1}
             {5}{{\textcolor{vu_orange_red}{5}}}{1}
             {6}{{\textcolor{vu_orange_red}{6}}}{1}
             {7}{{\textcolor{vu_orange_red}{7}}}{1}
             {8}{{\textcolor{vu_orange_red}{8}}}{1}
             {9}{{\textcolor{vu_orange_red}{9}}}{1},             
}

% Assembly style
\lstdefinestyle{asmstyle}{
    language=[x86masm]Assembler,
    basicstyle=\ttfamily\footnotesize,
    keywordstyle=\color{vu_blue}\bfseries,
    stringstyle=\color{red},
    commentstyle=\color{gray}\itshape,
    numberstyle=\tiny\color{darkgray},
    numbers=none,
    stepnumber=1,
    showstringspaces=false,
    tabsize=4,
    breaklines=true,
    breakatwhitespace=false,
    frame=single,
    backgroundcolor=\color{white},
    morekeywords={mov, add, sub, mul, div, push, pop, call, ret, jmp, cmp, test, and, or, xor, not, shr, shl, inc, dec, lea, nop, movabs, cmovne}, % instructions
    morekeywords=[2]{rdx, rip, rbp, rax, r12, r13, r14, r15, rbx, rcx, rsp}, % registers
    keywordstyle=[2]\color{vu_orange_red}\bfseries,
}

% microcode style
\lstdefinestyle{uCode}{
    language=C,
    basicstyle=\ttfamily\footnotesize,
    keywordstyle=\color{vu_blue}\bfseries,
    stringstyle=\color{red},
    commentstyle=\color{gray}\itshape,
    numberstyle=\tiny\color{darkgray},
    numbers=none,
    stepnumber=1,
    showstringspaces=false,
    tabsize=4,
    breaklines=true,
    breakatwhitespace=false,
    frame=single,
    backgroundcolor=\color{white},
    morekeywords={LDZX_DSZ64_ASZ32_SC1_DR, SUB_DSZ64_DRR, UJMPCC_DIRECT_NOTTAKEN_CONDZ_RI, END_SEQWORD, NOP},   % ucode instructions
    morekeywords=[2]{TMP4, TMP5, RDI, TMP6}, % ucode registers
    keywordstyle=[2]\color{vu_orange_red}\bfseries,
    morekeywords=[3]{END_SEQWORD}
}

% Coffin dark magic for overlays
\NewCoffin \CenteredCoffin
\ExplSyntaxOn
\NewDocumentCommand \GlobalSetHorizontalCoffin { m +m } {
    \hcoffin_gset:Nn{#1}{#2}
}
\NewDocumentCommand \GlobalClearCoffin { m } {
    \coffin_gclear:N{#1}
}
\ExplSyntaxOff
\AtBeginShipout{%
    \AtBeginShipoutUpperLeftForeground{
        \TypesetCoffin\CenteredCoffin[hc,vc](.5\pdfpagewidth,-.5\pdfpageheight)
        \GlobalClearCoffin\CenteredCoffin
    }
}


% Thickness of Progressbar (dummy thicc)
\makeatletter
\setlength{\metropolis@titleseparator@linewidth}{1.5pt}
\setlength{\metropolis@progressonsectionpage@linewidth}{1.5pt}
\setlength{\metropolis@progressinheadfoot@linewidth}{1.5pt}
\makeatother

\usepackage{xspace}
\newcommand{\themename}{\textbf{\textsc{metropolis}}\xspace}

\title[XM\_0123]{Thesis Title}
%\subtitle{Author Name}
\date{01.01.1970}
\author{Author Name}
\institute{Supervisor: Dr. Doctor \\ \\ VU Amsterdam - CSec}
\titlegraphic{\vspace{4.25cm}\hfill\includesvg[height=1cm]{images/VU-logo-nobg.svg}}

\begin{document}

% Set Colors
\setbeamercolor{frametitle}{bg=vu_blue, fg=white}
\setbeamercolor{normal text}{fg=special_gray}
\setbeamercolor{example text}{fg=special_gray}
\usebeamercolor[fg]{normal text}
\setbeamercolor{progress bar in head/foot}{fg=vu_orange_red}
\setbeamercolor{progress bar in section page}{fg=vu_orange_red}
\setbeamercolor{title separator}{fg=special_light_gray}
% Set Colors

\logo{\vspace{-0.6cm}\includesvg[height=0.5cm]{images/VU-logo-nobg.svg}\hspace{0.855\paperwidth}}

\maketitle

\begin{frame}[fragile]{My thesis in two sentences}
    Main contributions:
    \begin{itemize}
        \item First Sentence
        \item Second Sentence
    \end{itemize}
\end{frame}

\begin{frame}[fragile]{Slide with image}
    This is a text
    \begin{itemize}
        \item Bulletpoint 1
        \item Bulletpoint 2
        \item Bulletpoint 3
    \end{itemize}
    \begin{figure}
        \centering
        \includegraphics[width=0.3\textwidth]{example-image-a}
        \label{fig:fig_1}
    \end{figure}
\end{frame}

\begin{frame}[fragile]{General Slide with code}
    \begin{center}
            This is a text
    \end{center}

    \begin{columns}
        \begin{column}{0.50\textwidth}
\begin{lstlisting}[style=cstyle]
#include <stdio.h>

void c_function1(void) {
	printf("This is a function.\n");
}
\end{lstlisting}
        \end{column}
    
        \begin{column}{0.50\textwidth}
\begin{lstlisting}[style=cstyle]
#include <stdio.h>

void c_function2(void) {
	printf("This is a function.\n");
}
\end{lstlisting}
        \end{column}
    \end{columns}
\end{frame}

\begin{frame}[fragile]{}
    \section{Separator slide}
\end{frame}

\begin{frame}[fragile]{Slide with underlined text}
    \begin{columns}
        \begin{column}{0.55\textwidth}
            This is underlined \ulinegreen{green}
            \begin{itemize}
                \item Bulletpoint 1
                \item Bulletpoint 2
            \end{itemize}

            \vspace{0.5cm}
            This is underlined \ulinegrey{grey}
            \begin{itemize}
                \item Bulletpoint 1
                \item Bulletpoint 2
            \end{itemize}
        \end{column}

        \begin{column}{0.45\textwidth}
            \begin{figure}
                \centering
                \includegraphics[width=0.8\textwidth]{example-image-b}
                \label{fig:fig_2}
            \end{figure}
        \end{column}
    \end{columns}
\end{frame}


\begin{frame}[fragile]{}
    \section{Separator slide}
\end{frame}

\begin{frame}[fragile]{Slide with tikz drawing}
\tikzstyle{prep} = [rectangle, rounded corners, minimum width=\linewidth-7cm, minimum height=0.65cm, text centered, draw=black, fill=cyan!30]
\tikzstyle{process} = [rectangle, rounded corners, minimum width=\linewidth-7cm, minimum height=0.65cm, text centered, draw=black, fill=gray!30]
\tikzstyle{open} = [rectangle, rounded corners, minimum width=\linewidth-6cm, minimum height=0.65cm, text centered, draw=black, fill=red!30]
\tikzstyle{spec} = [rectangle, rounded corners, minimum width=\linewidth-8cm, minimum height=0.65cm, text centered, draw=black, fill=red!20]
\tikzstyle{decision} = [rectangle, rounded corners, minimum width=\linewidth-8cm, minimum height=0.65cm, text centered, draw=black, fill=orange!20]
\tikzstyle{win} = [rectangle, rounded corners, minimum width=\linewidth-8cm, minimum height=0.65cm, text centered, draw=black, fill=green!20]
\tikzstyle{arrow} = [thick,->,>=stealth]
\begin{figure}[h]
	\centering
	\begin{tikzpicture}[node distance=1.1cm]
		% Nodes
		%\node (box1) [process] {Fuzz target program};
		\node (box2) [process] {Mond};
		\node (box3) [open, below of=box2] {Bober};
		\node (box4) [spec, below of=box3] {Procione};
		\node (box5) [spec, below of=box4] {Word};
		\node (box6) [decision, below of=box5] {Text};
		\node (box7) [win, below of=box6] {Lizard};
		\node (box8) [open, below of=box7] {Montagne};

		% Arrows
		%\draw [arrow] (box1) -- (box2);
		\draw [arrow] (box2) -- (box3);

		% Speculation arrows
		\draw [arrow, dotted, color=black!60] (box3) -- (box4);
		\draw [arrow, dotted, color=black!60] (box4) -- (box5);
		\draw [arrow, dotted, color=black!60] (box5) -- (box6);
		\draw [arrow, dotted, color=black!60] (box6.east) -- ++(+0.5cm,0) node[anchor=north, pos=0.5]{Yes} |- (box7);
		\draw [arrow, dotted, color=black!60] (box6.west) -- ++(-0.5cm,0) node[anchor=north, pos=0.5]{No} |- (box4);

		% Speculation box
		\draw [dashed, color=red!30, thick] (box3.south east) ++(0,0) -- (box8.north east);
		\draw [dashed, color=red!30, thick] (box3.south west) ++(0,0) -- (box8.north west);
	\end{tikzpicture}
	\label{fig:s_overview_diagram}
\end{figure}
\end{frame}

\begin{frame}[fragile]{Slide with assembly}
    \begin{lstlisting}[style=asmstyle]
_start:
    int 0x80                    ; Interrupt goes brrrrr
    \end{lstlisting}
\end{frame}

\begin{frame}[fragile]{Slide containing one picture}
\begin{figure}
    \centering
    \includegraphics[height=0.5\paperwidth]{example-image-c}
\end{figure}
\end{frame}

\begin{frame}[fragile]{Slide with tables}
    \begin{columns}
        \begin{column}{0.5\textwidth}
            \begin{table}[h!]
                \centering
                \caption{Table}
                \begin{tabular}{|l|c|}
                    \hline
                    \textbf{Value 1} & \textbf{Value 2}\\
                    \hline
                    0x2a & 4242424242 \\
                    \hline
                    0x42 & 0o52 \\
                    \hline
                \end{tabular}
            \end{table}
        \end{column}

        \begin{column}{0.5\textwidth}
            \begin{table}[h!]
                \centering
                \caption{Table}
                \begin{tabular}{|l|c|}
                    \hline
                    \textbf{Value 1} & \textbf{Value 2}\\
                    \hline
                    0x2a & 42 \\
                    \hline
                    0x4242424242 & 0o52 \\
                    \hline
                \end{tabular}
            \end{table}
        \end{column}
    \end{columns}
\end{frame}

\begin{frame}[fragile]{Slide with microcode}
\begin{center}
    CISC CPUs are so bad, they actually use RISC instructions internally
\end{center}
\begin{lstlisting}[style=uCode]
/*
  Assumptions:
  - TMP6 contains value 0x2a
  - TMP4 contains value 42
*/
{
	LDZX_DSZ64_ASZ32_SC1_DR(TMP5, TMP4),
	SUB_DSZ64_DRR(TMP5, TMP6, TMP5),
	UJMPCC_DIRECT_NOTTAKEN_CONDZ_RI(TMP5, jump),
	END_SEQWORD
},
{
jump:
	DO_BONKY_MANEUVER(),
	NOP,
	NOP,
	END_SEQWORD
}
\end{lstlisting}
\end{frame}

\begin{frame}[fragile]{Slide with table and assembly listing}
    \begin{columns}
        \begin{column}{0.35\textwidth}
\begin{table}[h!]
    \centering
    \begin{tabular}{|l|r|}
        \hline
        \textbf{Farbe} & \textbf{Constellation}\\
        \hline
        0x4192529 & 124214\\
        \hline
        0x8f8dd828 & 213303\\
        \hline
        0x555a5a5a & 4343\\
        \hline
        0x7ffff7de & 42\\
        \hline              
    \end{tabular}
\end{table}
        \end{column}

        \begin{column}{0.625\textwidth}
\begin{lstlisting}[
    style=asmstyle,
    ]
  00000000004011c7 <otter>:
    [...]
    4011e3:       cmp    %rax,%rdx
    4011e6:       jae    401227 <vertigo>
    4011e8:       mov    -0x10(%rbp),%rax
    4011ec:       mov    (%rax),%rax
    4011ef:       lea    0x0(,%rax,8),%rdx
    4011f7:       lea    0x2e82(%rip),%rax
    4011fe:       mov    (%rdx,%rax,1),%rax
    [...]
    40121d:       cmp    %r13,%rax
    401220:       cmovne %r14,%r15
    401224:       jmp    *%r15
  
  0000000000401227 <vertigo>:
    401227:       nop      
    401228:       mov    -0x8(%rbp),%r12
    40122c:       leave           
    40122d:       ret   
\end{lstlisting}
        \end{column}
    \begin{column}{0.025\textwidth}
    \end{column}
    \end{columns}
\end{frame}

\begin{frame}[fragile]{}
    \section{Questions and Notes?}
\end{frame}

%%% Reset slide counter | Start Appendix and backup slides
\newcounter{finalframe}
\setcounter{finalframe}{\value{framenumber}}

\appendix

\begin{frame}[fragile]{Slide with coffin overlay}

This text contains a very wrong statement and should be marked thus

\GlobalSetHorizontalCoffin \CenteredCoffin {%
    \pause
    \begin{figure}
        \centering
        \includesvg[width=0.2\linewidth]{images/Red_X.svg}
    \end{figure}
    \onslide<1->% Reset things, so the following content isn't paused
}
\end{frame}

%%% End of Appendix Counter
\setcounter{framenumber}{\value{finalframe}}
%\appendix



\end{document}
